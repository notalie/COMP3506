\documentclass[oneside, a4paper]{article}
\usepackage[utf8]{inputenc}
\usepackage[english]{babel}
\usepackage[hypertexnames=false]{hyperref} 
\hypersetup{
    colorlinks=true,
    linkcolor=black,
    filecolor=magenta,      
    urlcolor=cyan,
}

\urlstyle{same}
\usepackage{textcomp}
\usepackage[utf8]{inputenc}
\usepackage{graphicx}
\usepackage{array}
\usepackage{soul}
\usepackage{amsmath}
\usepackage{forest}
\usepackage{mathtools}
\DeclarePairedDelimiter\ceil{\lceil}{\rceil}
\DeclarePairedDelimiter\floor{\lfloor}{\rfloor}


% Set spacing (i set it to 1.2x)
\renewcommand{\baselinestretch}{1}
% Indentation (set this to zero for normal prose)
\setlength{\parindent}{0em}
% Line breaking (spacing between paragraphs)
\setlength{\parskip}{0.5em}

% Use the whole page
\usepackage{geometry}
% Extra math glyphs
\usepackage{amsmath}
% Proper enumerate spacing
\usepackage{enumitem}
% More pleasing screen fonts
\usepackage{lmodern}
% Fancy headers
\usepackage{fancyhdr}
\usepackage{graphicx}
\usepackage{algpseudocode}
% Allows absolute positioning of images
\usepackage{float}
% \usepackage[section]{placeins}
% Set no separation
\setlist{noitemsep}
% Set margins to reasonable
\geometry{margin=2.5cm}
% Sets graphics path
\graphicspath{ {./images/} }
% Sets up fancy headers

\addto\captionsenglish{
}


\usepackage{listings}
\usepackage{color}

\pagestyle{plain}

\begin{document}


\definecolor{dkgreen}{rgb}{0,0.6,0}
\definecolor{gray}{rgb}{0.5,0.5,0.5}
\definecolor{mauve}{rgb}{0.58,0,0.82}

\lstset{frame=tb,
  language=Python,
  aboveskip=3mm,
  belowskip=3mm,
  showstringspaces=false,
  columns=flexible,
  basicstyle={\small\ttfamily},
  numbers=none,
  numberstyle=\tiny\color{gray},
  keywordstyle=\color{blue},
  commentstyle=\color{dkgreen},
  stringstyle=\color{mauve},
  breaklines=true,
  breakatwhitespace=true,
  tabsize=3
}

\pagestyle{fancy}
\fancyhf{}
\lhead{s4530974 - Homework Assignment 1}
\rhead{COMP3506}

\begin{titlepage}
\newgeometry{left=7.5cm} %defines the geometry for the titlepage
\noindent
\color{black}
\makebox[0pt][l]{\rule{1.3\textwidth}{1pt}}

{\Huge {Student - s4530974}}
\vskip\baselineskip
\noindent
{\huge{COMP3506 - Homework 1}}

\vskip\baselineskip
{\large {Semester 2 - 2020}}
\end{titlepage}

\newpage
\tableofcontents

\newpage

\setcounter{secnumdepth}{-1}
\section{Question 1}

\subsection{1a - The Running Time of $T_{odd}$}
If n is odd then the if statement skips to this section of the function which then becomes $T_{odd}$. 

\
\begin{algorithmic}[1]
    \Function{$T_{odd}$}{int n}
    \State $sum \gets 0$ [1]
    \While {$n > 0$} [?? + 1]
        \State $sum \gets sum + (n \& 1)$ [3] 
        \State $n \gets (n >> 1)$ [2]
    \EndWhile
    \State \Return $sum$ [1]
    \EndFunction
\end{algorithmic}

\subsubsection*{Solving the ?? Part}
The running of the while loop part of the function is determined by the value of n and its most significant bit, you can assume that this bit would determine the times that the while loop would execute within $T_{odd}$. 

The function cannot be $O(n)$ as the amount of while loop iterations does not linearly increase as the number increases. As seen below with the first 9 inputs, going from 5 to 7 does not increase the amount of while loops iterations by a constant factor but is actually the same.

\begin{center}
    \begin{tabular}{|c c|} 
    \hline
    Number & While Loop Iterations \\ [0.5ex] 
    \hline
    1 & 1 \\ 
    \hline
    3 & 2\\
    \hline
    5 & 3 \\
    \hline
    7 & 3\\
    \hline
    9 & 4 \\ 
    \hline
    11 & 4 \\ 
    \hline
    13 & 4 \\ 
    \hline
    15 & 4 \\ 
    \hline
    17 & 5 \\ 
    \hline
   \end{tabular}
\end{center}

Since the most significant bit is being used as an indicator of how many times the while loop executes, you can use a formula to find which power of 2 the most significant bit would lie on. This is done with the equation:

$$\ceil[\Bigg]{\frac{\log{(n)}}{\log(2)}} = \ceil{\log_2(n)}$$ 
Multiplying this equation by the amount of operations within the while loop and then adding 1 to the amount of iterations to account for the while loop check will give:

$$\ceil{\log_2(n)}+ 1$$ 

This equation was then multiplied by the number of primitive operations within the while loop which add up to be 5. This was then multiplied by the existing equation to give:

$$\ceil{5\log_2(n)} + 5$$ 

You also have to account for the assigning of sum and returning of sum at the beginning as well as the if statement. This adds on 2 extra to the constant at the end resulting in the ending formula of:

$$\ceil{5\log_2(n)} + 7$$ 

An assumption made was that the $n \& 1$ and $sum + (n \& 1)$ counted as 1 and 2 primitive operations each respectively which is why I counted the line with the assignment as 3.

\subsection{1b - Finding a suitable function for $O(n)$}

A suitable function $g(n)$ that would fit $T_{odd}$ such that $g(n) \epsilon O(n)$. Since we know that the runtime function can be described as $\ceil{5\log_2(n)} + 7$ we can start from there.

First, remove all constants and lower order terms which leaves us with $\ceil{\log_2(n)}$. 

Using the definition for $O(n)$ which is:
Given functions $f(n)$ and $g(n)$, we say that $f(n)$ is $O(g(n))$ if there are positive constants $c$ and $n_0$ such that 
$f(n) \leq c * g(n)$ for $n \geq n_0$

\begin{center}
    Suppose $g(n) = log(n), n_0 = 5, c = 8$

    Sub $c = 3$, $\ceil{\log_2(n)} \leq 8log(n)$

    Sub $n = 5$, $\ceil{\log_2(5)}  \leq 8log(n)$

    $3 \leq 5.59$ is true

    $\dot{.\hspace{.095in}.}\hspace{.5in} \ceil{\log_2(n)} \text{ }\epsilon \text{ } O(log(n))$ when $n \geq 5, c = 8$
\end{center}

 
\subsection{1c - Finding $\Omega$}
Using the definition for $O(n)$ which is:
Given functions $f(n)$ and $g(n)$, we say that $f(n)$ is $\Omega(g(n))$ if there are positive constants $c$ and $n_0$ such that $f(n) \geq c * g(n)$ for $n \geq n_0$

\begin{center}
    Suppose $g(n) = log(n), n_0 = 5, c = 3$
    Sub $c = 3$, $\ceil{\log_2(n)} \geq 3log(n)$

    Sub $n = 5$, $\ceil{\log_2(5)} \geq 3log(5)$

    $3 \geq 2.096...$

    $\dot{.\hspace{.095in}.}\hspace{.5in} \ceil{\log_2(n)} \text{  }\epsilon \text{ } \Omega(log(n))$ when $n \geq 5, c = 3$
\end{center}

In terms of $\Theta(n)$, I believe it does exist. This can be proved using the definition given as:
$f(n)$ is $\Theta(g(n))$ if there exist positive constants $c_1, c_2$ and $n_0$ such that $c_1 * g(n) \leq f(n) \leq c_2 * g(n)$ for all $n \geq n_0$.

Since $g(n)$ in this case would be $log(n)$ and $f(n)$ would be $log(n)$. Suppose $c_1 = 1$, $c_2 = 2$ and $n = 2$

$1 * log(n) \leq log(n) \leq 2 * log(n)$ 

$log(2) \leq log(2) \leq 2log(2)$, for all $n \geq 2$

\subsection{1d - Bounds for $T_{even}$}

\begin{algorithmic}
    \Function{$T_{even}$}{int n}
    \For {$i = 0$ to n} [n + 1]
        \For {$j = i$ to $n^2$} [$n^2$]
            \State {$sum \gets sum + i + j$} [3]
        \EndFor
    \EndFor
    \State \Return $sum$
    \EndFunction
\end{algorithmic}

$T_{even}$ has a runtime complexity of $(n+1)(n^2)(3) = 3n^3 + 3n^2$

I am assuming that the condition for the for loop is $i \leq n$ rather than $i < n$.  

The Upper Bound for $T_{even} \text{ } \epsilon \text{ } O(n^3)$ when $c = 1$ and $n \geq 1$

The Lower Bound for $T_{even} \text{ } \epsilon \text{ } \Omega(n)$ when $c = 1$ and $n \geq 1$ 

\subsection{1e}
The worst case will always occur within $T_{even}$ as it has a larger runtime complexity. Therefore, the Big-$\Omega$ of the worst case in the algorithm is $\Omega(n^3)$ occuring in $T_{even}$. The same logic can be applied vice-versa to $T_{odd}$ and as such, the Big-$O$ of the best case in the algorithm is $O(log(n))$ occuring in $T_{odd}$.

\subsection{1f}
The Big-$\Omega$ Bound for the algorithm is $\Omega(log(n))$ while the Big-$O$ Bound for the algorithm is $O(n^3)$. In terms of Big-$\Theta$, I believe that it does not exist as $\Omega(g(n)) \neq \Theta(g(n))$.


\subsection{1g}
My team mate is wrong as the best case does not define the bounds which are what Big-$\Omega$ and Big-$O$ are. A worst and best case scenario describe a singular scenario and are a singular point within the space of the algorithm where Big-$\Omega$ and Big-$O$ represent bound the lower and higher bounds respectively. They also define where worst and best cases can take place within the algorithm's asymptotic runtime.

\subsection{1h - Some $\Theta(f(n))$ Proof}
Statement - $\Theta(g(n))$ only exists when $O(g(n)) = \Omega(g(n))$. 

This can be proved using the definition given as:
$f(n)$ is $\Theta(g(n))$ if there exist positive constants $c_1, c_2$ and $n_0$ such that $c_1 * g(n) \leq f(n) \leq c_2 * g(n)$ for all $n \geq n_0$

Since $O(g(n)) = \Omega(g(n))$, for $\Theta(f(n))$ to exist, it must be $g(n)$ otherwise the definition does not hold. 

Therefore $\Theta(f(n))$ is $\Theta(g(n))$ when $O(g(n)) = \Omega(g(n))$.

\newpage

\setcounter{secnumdepth}{-1}
\section{Question 2}

\subsection{2a}
\begin{algorithmic}
    \Function{FINDPOSITIONRECURSE}{A[n], int low, int high}
        \If{$high \geq low$}
            \State $mid \gets low + (high - low) / 2$
            \If{mid == A[mid]}
                \State \Return true    
            \EndIf
                
            \If{$mid < A[mid]$}
                \State \Return FINDPOSITIONRECURSE(A, low, (mid - 1))
            \Else
                \State \Return FINDPOSITIONRECURSE(A, (mid + 1), high)
            \EndIf

        \Else 
            \State \Return false    
        \EndIf
        
    \EndFunction
\end{algorithmic}
\
\
\begin{algorithmic}
    \Function{FINDPOSITION}{A[n]}
        \State return FINDPOSITIONRECURSE(A, 0, n - 1)
    \EndFunction
\end{algorithmic}


\subsection{2b}
The steps taken within my function would take in the array and then call the recursive function FINDPOSITIONRECURSE low and high values set as 0 and (n - 1) respecitvely. The function would then begin searching to see if $A[i] == i$. The recursion would be as follows:

\begin{itemize}
    \item -1 \textrightarrow the function would calculate and set mid as 4, it would then check if $A[4] == 4$, which it is not. It would then check if $4 < A[4]$, which it is. It would the enter recursion and pass in low as the same value but change high to be that of (mid - 1). 
    \item 0 \textrightarrow the function would calculate and set mid as 1, it would then check if $A[1] == 1$, which it is not. It would then check to see if $1 < A[1]$, which it is not. It would then enter recursion again and pass in high as the same value and change low to be that of (mid + 1). 
    \item 2 \textrightarrow the function would calculate and set mid as 2, it would then check to see if $A[2] == 2$, which it is. The function would then exit recursion and return true.
\end{itemize}

\subsection{2c}
The worst case for my algorithm is where none of the values in A meet the conditions and the if high and low variables do not meet the requirements for the if statement $high \geq low$ and false is returned. The recurrence for this worse case would look like the following recurrence tree:

\begin{center}
    \begin{forest}
    for tree={
      draw,
      align=center
    }
    [$T(n) + 1$
      [$T(\frac{n}{2})$
        [$T(\frac{n}{4})$]
        [$T(\frac{n}{4})$
            [$T(\frac{n}{2^k})$]
        ]
      ]
      [$T(\frac{n}{2})$
        [$T(\frac{n}{4})$]
        [$T(\frac{n}{4})$]
      ]
    ]
  \end{forest}
\end{center}
  

When put into a formula, the recurrence equation would look like the following:

\begin{equation}
    \begin{array}{l}
        T(n) = T(\frac{n}{2}) ... + 1\\\\
        T(n) = T(\frac{n}{2}) + T(\frac{n}{4}) + T(\frac{n}{8}) ... T(\frac{n}{2^k}) + 1
      \end{array}
\end{equation}

The constant is 1 because that is what is used for the return statement out of recursion. 

Since the tree eventually ends as $T(\frac{n}{2^k})$, you could break this function down to 

Since we know that $T(1) = 1$ as given by the base case. We can do the following:
$$\frac{n}{2^k} = 1$$
$$n = 2^k$$
$$k = log_2n$$

Therefore $O(g(n))$ of this algorithm is $O(log(n))$.

\subsection{2d - Recurrences}

\subsubsection{2di}
Since we know that $O(log(n))$ and that 

\begin{equation}
    T(n)=\begin{cases}
        T(\frac{n}{2^k}) + 1, & \text{if $n>1$}.\\
        1, & \text{$n = 1$}.
    \end{cases}
\end{equation}

With $a = 1, b = 2^k, c = 1, g(n) = 1$. 

Since $g(n) \epsilon \Theta(n^d), d = 0$

Using the Master Theorem, since $a = {2^k}^0 = 1$ then $T(n) = \Theta(log(n))$. The algorithm is therefore $\Theta(log(n))$. 

\subsubsection{2dii}
\begin{center}
    $$T(n) = 5 * T (\frac{n}{3}) + n^2 + 2n$$

    Where $a = 5, b = 3, g(n) = n^2 + 2n$

    Since $g(n) \epsilon \Theta(n^d)$ 

    Then $g(n) \epsilon \Theta(n^2)$ and $d = 2$

    Sicne $5 < 3^2$, the $\Theta$ bounds for $T(n)$ is $\Theta(n^2)$ where 2 is d for when $T(1) = 100$.
\end{center}


\subsubsection{2diii}
Since n = k, the $\Theta$ is  $\Theta(1)$ for when $T(1) = 1$.

\subsection{2e}
\begin{algorithmic}
    \Function{FINDPOSITION}{A[n]}
        \State $low \gets 0$
        \State $high \gets (n - 1)$

        \While{$high \geq low$}

            \State $mid \gets low + (high - low) / 2$
            \If{mid == A[mid]}
                \State \Return true    
            \EndIf
                
            \If{$mid < A[mid]$}
                \State $high \gets (mid + 1)$
            \Else
                \State $low \gets (mid - 1)$
            \EndIf

        \EndWhile
        \State \Return false    
        
    \EndFunction
\end{algorithmic}

The runtime complexity of my second solution is within $O(log(n))$ time. This is because while there is no recursion within my program, it is still a binary search implementation which works by splitting the search size by 2 which eventually leads to $log_2(n)$ which gives the same complexity as the recursive one.

\subsection{2f}
For the consideration of larger inputs, I believe that the iterative method is better. This is because if there are more elements within A (as n increases), there runs the risk of getting a Stack Overflow due to a recursive depth being reached. This is mainly due to recursive methods relying on the stack which is ineffective for larger inputs.

In terms of space complexity, iterative methods rely on one stack instance which can get quite large. On the other hand as recursive methods continously create a new stack on each resursive call which takes up much more memory. This means that due to these multiple stacks, recursive methods will take up more space resulting in a slower program for larger inputs.

\subsubsection{References}
\href{https://medium.com/backticks-tildes/iteration-vs-recursion-c2017a483890}{Iteration vs Recursion}

\href{https://www.ideserve.co.in/learn/time-and-space-complexity-of-recursive-algorithms}{Time and Space Complexity of Recursive Algorithms}

\newpage
\setcounter{secnumdepth}{-1}
\section{Question 3}
\subsection{3b}
The Big-$O$ for my CartesianPlane class is $O(nm)$ where n is the length and m is the width of the plane.

\subsection{3c}
The cartesian plane still needs to reserve space for null elements. Additionally large elements means more wasted space. You could store pointers in the array instead of the actual objects but this adds another level of indirection.

\subsection{3d}
$n$ and $m$ represent the length and width of the CartesianPlane

\begin{tabular}{|c c p{0.7\textwidth}|} 
    \hline
    Method & Big-$O$ & Reasoning \\ [0.5ex] 
    \hline
    add & $O(1)$ & It takes constant time to add the value within the 2D array as you only need to get the indexes for the cartian plane \\ 
    \hline
    get & $O(1)$ & It takes constant time to access a variable within a 2D array as it only needs to index and return the value \\
    \hline
    remove & $O(1)$ &  It takes constant time to get the value from the 2D array and return a boolean depending if the value could or could not be returned \\
    \hline
    resize & $O(nm)$ & The for loops within the function increase in complexity as the length/width increase so the function has a complexity of $O(nm)$\\
    \hline
    clear & $O(nm)$ & The for loops for setting each value within the 2D array to null/empty values change in complexity with the changing of the length/width of the array \\ 
    \hline
\end{tabular}

\newpage
\setcounter{secnumdepth}{-1}
\section{Question 4}
\subsection{4a - My algorithm but in words}
Some things I would like to note about my algorithm that I would need to implement/consider:
\begin{itemize}
    \item The water source will always be below the corresponding x or y value when a drill breaks 
    \item It's probably best to start from the center
    \item When a drill breaks, you should always make sure to check if the plot is a water source or not in case that both conditions are met
    \item Once you have the water direction on one axis, you do not need to search that axis again
    \item Bounds for searching for the water source would need to be set and changed during the algorithm for an efficient function
\end{itemize}

\subsubsection{Description of my algorithm}
First I would set bounds $x_n, y_n$ which would represent a bound to search for the water source starting at 0 to n - 1. Once an axis (direction for the water source) has been found, these bounds would change values to indicate this.

Then I would begin my algorithm starting from the center of the plot and dig there (I would round up if the length/width of the plot was even). 

From there I would do the following checks:
\begin{enumerate}
    \item Check if any axis ($x_n$ or $y_n$) has already been found. 
    \begin{itemize}
        \item If it has, skip the checks for the already found axis broken
    \end{itemize}
    \item Check if the drill breaks
        \begin{itemize}
            \item Lower the bounds ($x_n$ or $y_n$) of the axis if the drill is broken. 
            \item Increase the bounds ($x_n$ or $y_n$) of the axis if the drill has not been broken
        \end{itemize}
    \item Check if a water source has been found
    \begin{itemize}
        \item Set the axis ($x_n$ or $y_n$) as found if the direction is known
    \end{itemize}
\end{enumerate}

I would then dig again from the center of the new bounds set until a water source is eventually found.

The Big-$O$ complexity on the amount of holes to dig would be $O(log_2(n))$ holes as the bounds decrease to half/the center of the grid which leads to a smaller bound with each search. This search is a binary search across two axis and as such, the Big-$O$ notation of my algorithm is $O(log(n))$. 

\subsection{4b - Minimal Algorithm}

Returns a true if the drilled plot is a water source and false if it is not.
\begin{algorithmic}[1]
    \Function{IsWaterSource}{G[n], int x, int y}
        \State \Return boolean isSource
    \EndFunction
\end{algorithmic}
\
Digs in the water hole in the x, y position and returns true if the drill is broken and false if it is not
\begin{algorithmic}[1]
    \Function{IsBroken}{int x, int y}
        \State \Return boolean pumpStatus
    \EndFunction
\end{algorithmic}

\
\begin{algorithmic}[1]
    \Function{FindWaterSourceMinimal}{G[n]}
    \State $x \gets 0$
    \State $y \gets 0$
    
    \While {$x < n$}
        \If {$isBroken(x, y)$}
            \State break
        \Else \State $x \gets x + 1$
        \EndIf
    \EndWhile
    
    \While {$y < n$}
        \If {$isBroken(x, y)$}
            \State break
        \Else \State $x \gets y + 1$
        \EndIf
    \EndWhile
    \State \Return (x, y)
    \EndFunction
\end{algorithmic}

The Big-$O$ bound for my function is $O(2n)$ where n is the length of the grid.

\end{document}