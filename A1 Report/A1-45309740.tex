\documentclass[oneside, a4paper]{article}
\usepackage[utf8]{inputenc}
\usepackage[english]{babel}
\usepackage[hypertexnames=false]{hyperref} 
\hypersetup{
    colorlinks=true,
    linkcolor=black,
    filecolor=magenta,      
    urlcolor=cyan,
}

\urlstyle{same}
\usepackage{textcomp}
\usepackage[utf8]{inputenc}
\usepackage{graphicx}
\usepackage{array}
\usepackage{soul}
\usepackage{amsmath}
\usepackage{forest}
\usepackage{mathtools}
\DeclarePairedDelimiter\ceil{\lceil}{\rceil}
\DeclarePairedDelimiter\floor{\lfloor}{\rfloor}


% Set spacing (i set it to 1.2x)
\renewcommand{\baselinestretch}{1}
% Indentation (set this to zero for normal prose)
\setlength{\parindent}{0em}
% Line breaking (spacing between paragraphs)
\setlength{\parskip}{0.5em}

% Use the whole page
\usepackage{geometry}
% Extra math glyphs
\usepackage{amsmath}
% Proper enumerate spacing
\usepackage{enumitem}
% More pleasing screen fonts
\usepackage{lmodern}
% Fancy headers
\usepackage{fancyhdr}
\usepackage{graphicx}
\usepackage{algpseudocode}
% Allows absolute positioning of images
\usepackage{float}
% \usepackage[section]{placeins}
% Set no separation
\setlist{noitemsep}
% Set margins to reasonable
\geometry{margin=2.5cm}
% Sets graphics path
\graphicspath{ {./images/} }
% Sets up fancy headers

\addto\captionsenglish{
}


\usepackage{listings}
\usepackage{color}

\pagestyle{plain}

\begin{document}


\definecolor{dkgreen}{rgb}{0,0.6,0}
\definecolor{gray}{rgb}{0.5,0.5,0.5}
\definecolor{mauve}{rgb}{0.58,0,0.82}

\lstset{frame=tb,
  language=Python,
  aboveskip=3mm,
  belowskip=3mm,
  showstringspaces=false,
  columns=flexible,
  basicstyle={\small\ttfamily},
  numbers=none,
  numberstyle=\tiny\color{gray},
  keywordstyle=\color{blue},
  commentstyle=\color{dkgreen},
  stringstyle=\color{mauve},
  breaklines=true,
  breakatwhitespace=true,
  tabsize=3
}

\pagestyle{fancy}
\fancyhf{}
\lhead{s4530974 - Homework Assignment 1}
\rhead{COMP3506}

\begin{titlepage}
\newgeometry{left=7.5cm} %defines the geometry for the titlepage
\noindent
\color{black}
\makebox[0pt][l]{\rule{1.3\textwidth}{1pt}}

{\Huge {Student - s4530974}}
\vskip\baselineskip
\noindent
{\huge{COMP3506 - Homework 1}}

\vskip\baselineskip
{\large {Semester 2 - 2020}}
\end{titlepage}

\newpage
\tableofcontents

\listoffigures

\newpage


\section*{Question 1}

\subsection*{1a - The Running Time of $T_{odd}$}
If n is odd then the if statement skips to this section of the function which then becomes $T_{odd}$. 

\
\begin{algorithmic}[1]
    \Function{$T_{odd}$}{int n}
    \State $sum \gets 0$ [1]
    \While {$n > 0$} [?? + 1]
        \State $sum \gets sum + (n \& 1)$ [3] 
        \State $n \gets (n >> 1)$ [2]
    \EndWhile
    \State \Return $sum$ [1]
    \EndFunction
\end{algorithmic}

\subsubsection*{Solving the ?? Part}
\hl{CHECK - I have no idea what this is uhhhh}

The running of the while loop part of the function is determined by the value of n and its most significant bit, you can assume that this bit would determine the times that the while loop would execute within $T_{odd}$. Since the most significant bit is being used as an indicator of how many times the while loop executes, you can use a formula to find which power of 2 the most significant bit would lie on. This is done with the equation 
$$\frac{\log{n}}{\log(2)}$$ 
Multiplying this equation by the amount of operations within the while loop and then adding 1 to the amount of iterations to account for the while loop check will give:

$$(\frac{\log{n}}{\log(2)} * 5) + 1$$

You also have to account for the assigning of sum and returning of sum at the beginning as well as the if statement. This adds on 2 extra to the constant at the end resulting in the ending formula of:

$$(\frac{\log{n}}{\log(2)} * 5) + 3$$

An assumption made was that the $n \& 1$ and $sum + (n \& 1)$ counted as 1 and 2 primitive operations each respectively which is why I counted the line with the assignment as 3. I am also assuming that $\frac{\log{n}}{\log(2)}$ gives an int rounded down.

\subsection*{1b - Finding a suitable function for $O(n)$}

A suitable function $g(n)$ that would fit $T_{odd}$ such that $g(n) \epsilon O(n)$. Since we know that the runtime function can be described as $(\frac{\log{n}}{\log(2)} * 5) + 3$ we can start from there.

First, remove all constants and lower order terms which leaves us with $\frac{\log{n}}{\log(2)}$. We then have to find the bound for $O(n)$ which can be defined by $f(n) \leq c * g(n)$. 

Suppose $g(n) = n^2, n_0 = 2, c = 3$

Then
 

\subsection*{1c - Finding $\Omega$}

The Lower Bound for $T_{odd} \epsilon \Omega(n)$ when $c = ??$ and $n_0 = ??$ 

\hl{Prove above thing}

In terms of $\Theta(n)$, I believe it does not exist as there is no guaranteed run time or within $T_{odd}$ that will be executed. It also does not exist as the upper and lower bounds are too small for $\Theta(n)$ to be defined.

\subsection*{1d - Bounds for $T_{even}$}

\begin{algorithmic}
    \Function{$T_{even}$}{int n}
    \For {$i = 0$ to n} [n-1]
        \For {$j = i$ to $n^2$} [$n^2 - i$]
            \State {$sum \gets sum + i + j$} [3]
        \EndFor
    \EndFor
    \State \Return $sum$
    \EndFunction
\end{algorithmic}

The Upper Bound for $T_{even} \epsilon O(n^3)$ when $c = ??$ and $n_0 = ??$ 

The Lower Bound for $T_{even} \epsilon \Omega(n)$ when $c = ??$ and $n_0 = ??$ 

\subsection*{1e}
The best case for this algorithm would be if n = 1. The run time of this would be 

The worst case for this algorithm would be contained within $T_{even}$ and would be $O(n^3)$

\subsection*{1f}
\subsection*{1g}
\subsection*{1h}

\newpage
\section*{Question 2}

\subsection*{2a}
\begin{algorithmic}
    \Function{FINDPOSITIONRECURSE}{A[n], int low, int high}
        \If{$high \geq low$}
            \State $mid \gets low + (high - low) / 2$
            \If{mid == A[mid]}
                \State \Return true    
            \EndIf
                
            \If{$mid < A[mid]$}
                \State \Return FINDPOSITIONRECURSE(A, low, (mid - 1))
            \Else
                \State \Return FINDPOSITIONRECURSE(A, (mid + 1), high)
            \EndIf

        \Else 
            \State \Return false    
        \EndIf
        
    \EndFunction
\end{algorithmic}
\
\
\begin{algorithmic}
    \Function{FINDPOSITION}{A[n]}
        \State return FINDPOSITIONRECURSE(A, 0, n - 1)
    \EndFunction
\end{algorithmic}


\subsection*{2b}
The steps taken within my function would take in the array and then call the recursive function FINDPOSITIONRECURSE low and high values set as 0 and n - 1 respecitvely. The function would then begin searching to see if $A[i] == i$. The recursion would be as follows:

\begin{itemize}
    \item -1 \textrightarrow the function would calculate and set mid as 4, it would then check if A[4] == 4, which it is not. It would then check if 4 < A[4], which it is. It would the enter recursion and pass in low as the same value but change high to be that of (mid - 1). 
    \item 0 \textrightarrow the function would calculate and set mid as 1, it would then check if A[1] == 1, which it is not. It would then check to see if 1 < A[1], which it is not. It would then enter recursion again and pass in high as the same value and change low to be that of (mid + 1). 
    \item 2 \textrightarrow the function would calculate and set mid as 2, it would then check to see if A[2] == 2, which it is. The function would then exit recursion and return true.
\end{itemize}

\subsection*{2c}
The worst case for my algorithm is where none of the values in A meet the conditions and the if high and low variables do not meet the requirements for the if statement $high \geq low$ and false is returned. The recurrence for this worse case would look like the following recurrence tree:

\begin{center}
    \begin{forest}
    for tree={
      draw,
      align=center
    }
    [$T(n) + 1$
      [$T(\frac{n}{2})$
        [$T(\frac{n}{4})$]
        [$T(\frac{n}{4})$
            [$T(\frac{n}{2^n})$]
        ]
      ]
      [$T(\frac{n}{2})$
        [$T(\frac{n}{4})$]
        [$T(\frac{n}{4})$]
      ]
    ]
  \end{forest}
\end{center}
  

When put into a formula, the recurrence equation would look like the following:

\begin{equation}
    \begin{array}{l}
        T(n) = T(\frac{n}{2}) ... + 1\\\\
        T(n) = T(\frac{n}{2}) + T(\frac{n}{4}) + T(\frac{n}{8}) ... T(\frac{n}{2^k}) + 1
      \end{array}
\end{equation}

The constant is 1 because that is what is used for the return statement out of recursion. 

Since the tree eventually ends as $T(\frac{n}{2^k})$, you could break this function down to 

Since we know that $T(1) = 1$ as given by the base case. We can do the following:
$$\frac{n}{2^k} = 1$$
$$n = 2^k$$
$$k = log_2n$$

Therefore $O(g(n))$ of this algorithm is $O(log(n))$.

\subsection*{2d}

\subsubsection*{2di}
Since we know that $O(log(n))$ and that 

\begin{equation}
    T(n)=\begin{cases}
        T(\frac{n}{2^k}) + 1, & \text{if $n>1$}.\\
        1, & \text{$n = 1$}.
    \end{cases}
\end{equation}

With $a = 1, b = 2^k, c = 1, g(n) = 1$. 

Since $g(n) \epsilon \Theta(n^d), d = 0$

Using the Master Theorem, since $a = {2^k}^0 = 1$ then $T(n) = \Theta(log(n))$. The algorithm is therefore $\Theta(log(n))$. 

\subsubsection*{2dii}
\begin{center}
    $$T(n) = 5 * T (\frac{n}{3}) + n^2 + 2n$$

    Where $a = 5, b = 3, g(n) = n^2 + 2n$

    Since $g(n) \epsilon \Theta(n^d)$ 

    Then $g(n) \epsilon \Theta(n^2)$ and $d = 2$

    Sicne $5 < 3^2$, the $\Theta$ bounds for $T(n)$ is $\Theta(n^2)$ where 2 is d for when $T(1) = 100$.
\end{center}


\subsubsection*{2diii}
Since n = k, the $\Theta$ is  $\Theta(1)$ for when $T(1) = 1$.

\subsection*{2e}
\begin{algorithmic}
    \Function{FINDPOSITION}{A[n]}
        \State $low \gets 0$
        \State $high \gets (n - 1)$

        \While{$high \geq low$}

            \State $mid \gets low + (high - low) / 2$
            \If{mid == A[mid]}
                \State \Return true    
            \EndIf
                
            \If{$mid < A[mid]$}
                \State $high \gets (mid + 1)$
            \Else
                \State $low \gets (mid - 1)$
            \EndIf

        \EndWhile
        \State \Return false    
        
    \EndFunction
\end{algorithmic}

The runtime complexity of my second solution is within $O(log(n))$ time. This is because while there is no recursion within my program, it is still a binary search implementation which works by splitting search size by 2 which eventually 

\subsection*{2f}
I believe that the better function to run within Java will be my iterative one. This is because if there are more elements within A, there runs the risk of getting a Stack Overflow during the program. 


\section*{Question 3}

\end{document}